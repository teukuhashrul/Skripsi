%versi 2 (8-10-2016) 
\chapter{Pendahuluan}
\label{chap:intro}
   
\section{Latar Belakang}
\label{sec:label}

Film merupakan media komunikasi yang bersifat audio visual untuk menyampaikan suatu pesan kepada penontonnya. Keberadaan film membuat masyarakat menjadikan film sebagai media hiburan. Beragam cara dapat dilakukan untuk menikmati sebuah film yaitu datang ke bioskop, membeli kaset DVD dan \textit{streaming} menggunakan aplikasi \textit{desktop} dan \textit{smartphone}.

Film yang dibuat ada karena ada kumpulan orang di balik layar yang bekerja untuk membuatnya. Terdapat beragam perusahaan produksi film yang berlomba-lomba untuk membuat film yang dapat memperoleh keuntungan maksimum. Dengan menciptakan film yang sesuai dengan keinginan penontonnya,maka peluang keuntungan yang diperoleh pun akan semakin meningkat dan dapat menutup \textit{budget} yang digunakan sebelumnya untuk biaya produksi.
	
Berdasarkan penelitian analisis data film yang ada sebelumnya pada \textit{kaggle}, data film memiliki beberapa atribut umum yaitu judul,\textit{genre}, \textit{rating}, keuntungan,\textit{ budget}, nilai \textit{review} dari situs internet dan aktor yang terlibat dan lama tayang. Data film yang ada digunakan untuk membantu menganalisis sifat dari film. Penelitian yang dilakukan adalah analisis prediksi nilai IMDB \textit{score}, yaitu situs yang berisi data film. 	
	

\textit{Genre} adalah sebutan untuk membedakan berbagai jenis film. Sebuah film memiliki satu atau beragam \textit{genre}. Terdapat banyak film yang menggunakan kombinasi dari beberapa \textit{genre}.  \textit{Genre} yang ada berupa \textit{action, adventure, animation ,drama ,comedy ,horror ,romance} dan lain-lain. 

Terdapat banyak kemungkinan faktor yang dapat dijadikan sebuah film dapat memperoleh keuntungan maksimum. Faktor kesuksesan film berupa faktor \textit{rating} dari situs \textit{review} film, nama aktor yang terlibat, nama sutradara yang terlibat dan jumlah \textit{budget} yang dikeluarkan. Salah satu faktor dan kombinasi beberapa faktor dapat memengaruhi kesuksesan film.
	
Berdasarkan uraian diatas, akan dilakukan sebuah penelitian mengenai data film. Penelitian ini adalah analisis kesuksesan film dengan \textit{data mining} untuk memperoleh faktor-faktor yang ada dapat memprediksi  kesuksesan sebuah film. Dari faktor yang diperoleh, maka akan diprediksi \textit{revenue}/pendapatan sebuah film berdasarkan data film yang sudah ada sebelumnya. 
	
Pada penelitian ini dibuat sekumpulan perangkat lunak yang digunakan untuk mengumpulkan, membersihkan data, analisis ,pembuatan model, evaluasi kerja model dan visualisasi data. Perangkat lunak yang dibuat akan membantu menganalisis data film yang digunakan. Pembuatan perangkat lunak akan menggunakan bahasa pemrograman \textit{Python} dan memanfaatkan beberapa \textit{library} dari \textit{Python}.\textit{Pandas} digunakan untuk integrasi data. \textit{Sci-kit learn} digunakan untuk implementasi regresi, teknik \textit{clustering} dan \textit{classification} untuk memprediksi keuntungan sebuah film. Penelitian ini akan melakukan eksperimen untuk membandingkan beberapa metode \textit{machine learning} dalam memprediksi kesuksesan sebuah film.


\section{Rumusan Masalah}
\label{sec:rumusan}
Berkaitan dengan identifikasi masalah yang ada pada deskripsi di atas, masalah-masalah yang ada dapat dirumuskan sebagai berikut. 

\begin{itemize}
\item Apa saja faktor yang dapat digunakan untuk menentukan kesuksesan sebuah film ?
\item Bagaimana langkah dalam melakukan analisis kesuksesan film dengan \textit{data mining} ? 
\item Bagaimana hasil pengujian pada penelitian ini ?
\end{itemize}


\section{Tujuan}
\label{sec:tujuan}
Tujuan yang ingin dicapai dari penelitian ini adalah:  

\begin{itemize}
\item Mengeksplorasi data yang dikumpulkan
\item Membuat perangkat lunak yang dapat menggunakan metode \textit{data mining} untuk melakukan analisis faktor-faktor yang dapat berpengaruh pada kesuksesan film
\item Menguji metode-metode yang digunakan pada penelitian ini
\end{itemize} 

\section{Batasan Masalah}
\label{sec:batasan} 
Pelaksanaan penelitian ini permasalahannya dibatasi pada: 

\begin{enumerate}
\item Dataset yang digunakan pada penelitian berasal dari situs penyedia dataset seperti \textit{Kaggle}
\item Data yang digunakan merupakan data IMDB dari tahun 2006 sampai 2016
\item Penelitian ini akan membuat mengimplementasikan tahapan \textit{data mining} dengan memanfaatkan \textit{library} dari \textit{Python}
\item Penelitian ini tidak membuat antarmuka perangkat lunak dalam memrediksi kesuksesan film sehingga penjabaran dalam penelitian ini menggunakan visualisasi data 

\end{enumerate}



\section{Metodologi}
\label{sec:metlit}
Langkah-langkah yang akan dilakukan dalam melakukan penelitian ini, yaitu: 
\begin{enumerate}
\item Melakukan studi literatur dengan mencari jurnal, \textit{paper} mengenai penelitian sejenis dari berbagai sumber untuk membantu penulis dalam menulis 
\item Melakukan studi literatur langkah-langkah teknik \textit{data mining} untuk memahami konsep
\item Melakukan studi literatur mengenai metode-metode \textit{machine learning} yaitu regresi, \textit{clustering} dan \textit{classification} yang relevan
\item Melakukan studi literatur mengenai teori dan implementasi visualisasi data seperti \textit{histogram,scatter plot, box plot} untuk membantu mengetahui sifat data yang dikumpulkan menggunakan \textit{matplotlib}
\item Melakukan penelitian sejenis mengenai industri perfilman untuk mengetahui relevansi antar faktor yang ada 
\item Mempelajari bahasa pemrograman \textit{Python} dan beberapa \textit{library} dari \textit{Python}  seperti \textit{Pandas},\textit{Sci-Kit learn} dan \textit{matplotlib}
\item Mencari sumber data yang relevan untuk melakukan pengumpulan data dari situs \textit{review} film dan media sosial
\item Melakukan integrasi data dari sumber yang digunakan
\item Melakukan eksplorasi untuk menemukan sifat data
\item Melakukan analisis data secara statistik  dengan teknik visualisasi data 
\item Menerapkan metode-metode \textit{machine learning} regresi, \textit{clustering} dan \textit{classification}
\item Melakukan pengujian dan eksperimen  
\item Menulis dokumen skripsi
\end{enumerate}

\section{Sistematika Pembahasan}
\label{sec:sispem}
Sistematika penulisan ini berguna untuk memberikan gambaran secara umum mengenai penelitian
yang akan dibuat. Berikut ini adalah uraian dari sistematika pembahasan :  

\begin{itemize}
\item Bab 1. Pendahuluan, membahas tentang latar belakang, rumusan masalah, tujuan penelitian, batasan masalah, metode penelitian dan sistematika pembahasan mengenai skripsi

\item Bab 2. Landasan Teori, membahas pengertian \textit{data mining}, langkah-langkah \textit{data mining}, algoritma \textit{Machine Learning}, visualisasi data dan \textit{web scraping}

\item Bab 3. Analisis, membahas tentang bagaimana implementasi teori yang dijelaskan pada bab sebelumnya. Pada bab ini juga menjelaskan implementasi menggunakan \textit{library Python}. 

\item Bab 4. Implementasi, membahas tentang deskripsi \textit{dataset}, hasil analisis data, pengujian prediksi dan interpretasi pola menarik dari visualisasi. 

\item Bab 5. Kesimpulan dan Saran 
membahas kesimpulan yang diperoleh setelah melakukan analisis data serta saran yang dapat diberikan untuk pengembangan lebih lanjut tentang analisis data film 

\end{itemize}

